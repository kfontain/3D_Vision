\documentclass[a4paper]{article}

%% Language and font encodings
\usepackage[frenchb]{babel}
\usepackage[utf8x]{inputenc}
\usepackage[T1]{fontenc}
\usepackage{minted} %compiler avec la commande -shell-escape
\usepackage{graphicx}

%% Todo List
\usepackage{enumitem,amssymb}
\newlist{todolist}{itemize}{2}
\setlist[todolist]{label=$\square$}
\usepackage{pifont}
\newcommand{\cmark}{\ding{51}}%
\newcommand{\xmark}{\ding{55}}%
\newcommand{\done}{\rlap{$\square$}{\raisebox{2pt}{\large\hspace{1pt}\cmark}}%
\hspace{-2.5pt}}
\newcommand{\wontfix}{\rlap{$\square$}{\large\hspace{1pt}\xmark}}

%% Sets page size and margins
\usepackage[a4paper,top=3cm,bottom=2cm,left=3cm,right=3cm,marginparwidth=1.75cm]{geometry}
\setlength{\parskip}{.5em}

\newcommand{\HRule}{\rule{\linewidth}{0.5mm}}

%-------------------------------------------------------------------------------
% TITLE PAGE
%-------------------------------------------------------------------------------

\title
{
	\LARGE{Projet Technologique}
	\HRule \\ [0.5cm]
	\LARGE \textbf{\uppercase{Vision Stéréoscopique}}
	\HRule \\ [0.5cm]
}

\author{Geoffrey MEILHAN \\ Mohamed ALAMI \\ Kenji FONTAINE}

\begin{document}

\null  % Empty line
\nointerlineskip  % No skip for prev line
\vfill
\let\snewpage \newpage
\let\newpage \relax
\maketitle
\let \newpage \snewpage
\vfill
\break % page break

%-------------------------------------------------------------------------------
% Table of Contents
%-------------------------------------------------------------------------------

\tableofcontents
\newpage

%-------------------------------------------------------------------------------
% Introduction
%-------------------------------------------------------------------------------

\section{Description du projet}
La stéréoscopie est un ensemble de techniques visant à créer ou améliorer la
perception de relief à partir de deux images planes.

Le projet consiste à implémenter un module permettant d'évaluer une distance à
partir de deux images planes en entrée.

\section{Domaine : Vision stéréoscopique}

I tried so hard

\section{Cahier des charges}

And got so far

\section{Architecture du code}

But in the end

\section{Tests}

It doesn't even matter

\section{Conclusion}

OPPA GANGNAM STYLE

%-------------------------------------------------------------------------------
% Architecture
%-------------------------------------------------------------------------------

\end{document}
