\documentclass[a4paper]{article}

%% Language and font encodings
\usepackage[frenchb]{babel}
\usepackage[utf8x]{inputenc}
\usepackage[T1]{fontenc}
\usepackage{minted} %compiler avec la commande -shell-escape
\usepackage{graphicx}

%% Todo List
\usepackage{enumitem,amssymb}
\newlist{todolist}{itemize}{2}
\setlist[todolist]{label=$\square$}
\usepackage{pifont}
\newcommand{\cmark}{\ding{51}}%
\newcommand{\xmark}{\ding{55}}%
\newcommand{\done}{\rlap{$\square$}{\raisebox{2pt}{\large\hspace{1pt}\cmark}}%
\hspace{-2.5pt}}
\newcommand{\wontfix}{\rlap{$\square$}{\large\hspace{1pt}\xmark}}

%% Sets page size and margins
\usepackage[a4paper,top=3cm,bottom=2cm,left=3cm,right=3cm,marginparwidth=1.75cm]{geometry}
\setlength{\parskip}{.5em}

\newcommand{\HRule}{\rule{\linewidth}{0.5mm}}

%-------------------------------------------------------------------------------
% TITLE PAGE
%-------------------------------------------------------------------------------

\title
{
	\LARGE{Projet Technologique}
	\HRule \\ [0.5cm]
	\LARGE \textbf{\uppercase{Vision Stéréoscopique}}
	\HRule \\ [0.5cm]
}

\author{Geoffrey MEILHAN \\ Mohamed ALAMI \\ Kenji FONTAINE}

\begin{document}

\null  % Empty line
\nointerlineskip  % No skip for prev line
\vfill
\let\snewpage \newpage
\let\newpage \relax
\maketitle
\let \newpage \snewpage
\vfill
\break % page break

%-------------------------------------------------------------------------------
% Table of Contents
%-------------------------------------------------------------------------------

\tableofcontents
\newpage

%-------------------------------------------------------------------------------
% Introduction
%-------------------------------------------------------------------------------

\section{Description du projet}
La stéréoscopie est un ensemble de techniques visant à créer ou améliorer la
perception de relief à partir de deux images planes.

Le projet consiste à développer un module permettant d'évaluer une distance à
partir de deux images planes en entrée. Ce projet prendra la forme d'une implémentation
du module sur un système mobile à roues. L'objectif final étant de concevoir un robot
suiveur, capable de suivre une personne à une distance donnée.

%-------------------------------------------------------------------------------
% Domaine
%-------------------------------------------------------------------------------

\section{Domaine : Vision stéréoscopique}

Aujourd'hui devenu peu coûteux et peu encombrant, les systèmes de vision
stéréoscopiques sont de plus en plus répandus. \\
On trouve de nombreux domaines d'application dont les suivants :
\begin{itemize}
	\item Système de freinage automatique chez les voitures (Toyota par exemple).
	\item Détection d'obstacle chez les voitures autonomes.
	\item Reconnaissance d'objets chez les robots.
	\item La réalité virtuelle
\end{itemize}

%-------------------------------------------------------------------------------
% Cahier des charges
%-------------------------------------------------------------------------------

\section{Cahier des charges}

And got so far

%-------------------------------------------------------------------------------
% Architecture du code QT CV
%-------------------------------------------------------------------------------

\section{Architecture du code : partie QT et OpenCV}

%%%%% CONVERT %%%%%
\subsubsection*{convert.cpp convert.h}

Ces fichiers permettent de convertir des images stockées sous le format d'openCV
(cv::Mat) en image sous format QT(QImage) et inversement. Il existe deux fonctions
principales : \textbf{mat2QImage} et \textbf{qImage2Mat}.

%%%%% EDIT %%%%%
\subsubsection*{edit.cpp edit.h}

Ces fichiers forment le "core" de notre module. Sont incluent la plupart des
algorithmes utilisés lors des calculs de carte de disparité ou de profondeur,
de détection de bords, de détections de points d'intérêt, etc...


\textbf{split :} Cette fonction permet de séparer verticalement une image en
entrée, en deux images de même taille. Elle prend en entrée une image de type
QImage. Cette QImage sera coupée en deux et chaque partie ainsi obtenue sera
respectivement stockée dans les cv::Mat dont les adresse sont données en paramètres.


\textbf{sobel :} Cette fonction permet de détecter les bords d'une image. Elle
prend en entrée une image de type cv::Mat. la fonction va effectuer sur l'image
source une détection de bords, puis stockera le résultat ainsi obtenu dans une
autre image cv::Mat dont l'adresse est donnée en tant que paramètre.


\textbf{surf et surfmatch :} Ces deux fonctions permettent d'effectuer une détection
de points d'intérêts. La fonction \textbf{surf} prend une seule image de type
cv::Mat en entrée. Elle va détecter les points d'intérêt sur l'image en entrée puis
les mettre en valeur. Le résultat obtenu sera stockée dans une autre image cv::Mat
dont l'adresse est donnée en tant que paramètre.


La fonction \textbf{surfMath} fait la même chose mais avec 2 images en entrée,
toujours de type cv::Mat. Elle va, en plus de la détection de points d'intérêts,
effectuer une mise en correspondance des points d'intérêts entre eux. Le résultat
obtenu sera stocké dans une cv::Mat de destination dont l'adresse est en paramètre
de la fonction.


\textbf{dispMap :} Cette fonction permet de calculer une carte de disparité à
partir de deux images cv::Mat en entrée. Elle fait appel à l'algorithme
\textbf{StereoBM} de la bibliothèque \textbf{OpenCV}. La carte de disparité ainsi
obtenue est stockée dans une image cv::Mat dont l'adresse est donnée en paramètre.

%%%%% MAIN %%%%%
\subsubsection*{main.cpp}

Il s'agit du code exécuté lors du lancement du programme. On distingue 2 parties.
Une première partie avec GUI correspond au code généré par défaut par QT.
Elle permet de lancer le programme avec l'interface graphique.

\begin{minted}{cpp}
int main(int argc, char *argv[])
{
	QApplication a(argc, argv);
	MainWindow w;
	w.show();

	return a.exec();
}
\end{minted}
%-------------------------------------------------------------------------------
% Architecture du code Unity
%-------------------------------------------------------------------------------

\section{Architecture du code : partie Unity}

But in the end

%-------------------------------------------------------------------------------
% Tests
%-------------------------------------------------------------------------------

\section{Tests}

- ouvrir une image
- détection de bords
- split
- depth/disp map
- unity (scripts)
-

It doesn't even matter

%-------------------------------------------------------------------------------
% Conclusion
%-------------------------------------------------------------------------------

\section{Conclusion}

OPPA GANGNAM STYLE

%-------------------------------------------------------------------------------
% In the end
%-------------------------------------------------------------------------------

\end{document}
